% !TEX root = note.tex
\documentclass[5pt,a4paper]{article}
\usepackage[T2A]{fontenc}
\usepackage[utf8]{inputenc}
\usepackage[russian,english]{babel}
%\usepackage[showframe]{geometry}
\usepackage[showcrop,includeheadfoot=true, hmargin=0mm, vmargin=0mm, top=0mm]{geometry}
\geometry{papersize={50. cm,25.4 cm}}

\usepackage{caption}
\usepackage{subcaption}
\usepackage{graphicx}
\usepackage{mathtools}
%\usepackage{biblatex}
%\addbibresource{sample.bib}
\usepackage[nottoc]{tocbibind}
\usepackage[toc,page]{appendix}
\usepackage{authblk}
\usepackage[8pt]{extsizes}

\topmargin -0.1in
\hoffset -0.30in
\textwidth 6.7in
\textheight 9.3in
\oddsidemargin 0.25in
\evensidemargin 0.25in

\setcounter{secnumdepth}{0}
\graphicspath{{./pdf-files/}{./pdfs_1/}}



\begin{document}
{\huge

\input{"yields.tex"}
\pagebreak 
\input{"syst.tex"}
\pagebreak 
\input{"dsyst_nuEleElastic.tex"}
\pagebreak 
\input{"dsyst_otherBG.tex"}
\pagebreak 
\input{"dsyst_all_bg.tex"}

\begin{equation}
\begin{split}
L(\boldsymbol{n},\boldsymbol{\theta^0}|\mu_\text{sig},\boldsymbol{b},\boldsymbol{\theta}) & = P_{SR} \times C_\text{syst} = \\
& = \prod_{i} P(n_i|\lambda_i(\mu_\text{sig},\boldsymbol{b},\boldsymbol{\theta})) \times C_\text{syst}(\boldsymbol{\theta^0},\boldsymbol{\theta})
\end{split}
\end{equation}


}
\end{document}
